\documentclass[11pt]{article} % auch Schriftgröße des Dokuments

%%% Packages %%%
\usepackage[ngerman]{babel} % Anpassungen für Deutsch
\usepackage[utf8]{inputenc} % Eingabekodierung
\usepackage[T1]{fontenc} % Ausgabekodierung
\usepackage{graphicx} % Bilder
\usepackage{stackengine} % angenehme Unterschriftenlinien
\usepackage{csquotes} % automatische deutsche Anführungszeichen im Quellenverzeichnis
\usepackage[textsize=tiny]{todonotes} % in-Document-Kommentare
\usepackage{outlines} % bessere Auflistungen

\usepackage{setspace} % Zeilenabstände definieren
\setstretch{1.5}
\setlength{\footnotesep}{\baselineskip}

\usepackage[  % Hyperlinks (auch Inhaltsverzeichnis)
    colorlinks,
    citecolor=black,
    filecolor=black,
    linkcolor=black,
    urlcolor=blue
]{hyperref}

\usepackage[ % Quellenverzeichnis
    backend=biber,
    style=apa,
    sorting=nyt
]{biblatex}

\usepackage[ % Seitenkonfiguration
    a4paper,
    top=2.5cm,
    bottom=2cm,
    left=4cm,
    right=2cm
]{geometry}

%%% Metadaten %%%
\title{
    <<Art der Arbeit>>\par
    zur Vorlesung \glqq{}<<Vorlesung>>\grqq{} im <<Semester>>\par
    an der Leibniz-Fachhochschule Hannover\par\bigskip

    \textbf{<<Titel der Arbeit>>}
}
\author{
  <<Name des Autors>> \small(M.-Nr. <<Martikelnummer>>)\\
  Studiengang: <<Studiengangsname>> (<<Abschlussname (B.Sc., ...)>>)\\[1em]
  <<Geprüft/Betreut/...>> von: <<Betreuungsperson>>\\
  << - ODER für Abschlussarbeiten - >>\\
  Erstgutachter<<in>>: <<ErstgutachterIn>>\\
  Zweitgutachter<<in>>: <<ZweitgutachterIn>>
}
\addbibresource{sources.bib}

%%% Dokumenteneinstellungen %%%
\setlength{\parindent}{0px} % Verhindert Einrückungen nach erstem Absatz im Kapitel
\renewcommand\familydefault{\sfdefault}  % Sans-Serif
\reversemarginpar % Kommentare auf der linken Seite
\renewcommand{\footnotesize}{\fontsize{9pt}{11pt}\selectfont} % Schriftgröße Fußnoten (und Zeilenabstand dieser)

\newcommand{\sigline}[2]{ % Marko für angenehme Unterschriftenlinien
  \begin{minipage}{#1}
    \centering
    \rule{\textwidth}{0.4pt}\\
    \small #2
  \end{minipage}
}

\sloppy % Bessere autom. Zeilenumbrüche

%%%%%%%%%%%%%%%%%%%%%%%%%%%%%%%%%%%%%%%%%%%%%%%%%%%%%%%%%%%%%%%%

\begin{document}
\pagenumbering{roman}

\maketitle

<<weitere Anmerkungen (Aufgabenstellungen etc.)>>

\newpage
\tableofcontents

\newpage
\pagenumbering{arabic}

\section{Kapitel}

Damit ein \textit{Compiler} deinen Code lesen kann, muss er nach \textcite[12]{nesbo_minnesota_2025} gewisse \textbf{Qualitätsanforderungen} erfüllen.
Verdeutlicht wird dieses Verhalten mit folgender Formel:
\begin{equation}
    \sum_{i=1}^{n} i = \frac{n(n+1)}{2}
\end{equation}

Damit diese Formel gilt, muss die Summe der Tages-, Monats- und Jahreszahl mit 3 multipliziert werden\footnote{Während eines Schaltjahres muss dieses Ergebnis noch mit 2 subtrahiert werden} und im Anschluss durch 56 \textbf{ohne Rest} teilbar sein. Ist dieses Kriterium \underline{nicht} erfüllt, muss man sich eine Wildkatze anschaffen, sie zähmen und anschließend mit insgesamt 15 Dosen würziger Lachspastete füttern. Ist das erledigt, kann man diese Formel zur Verhaltensanalyse verwenden: $ a + b + 2c^3 $, wobei $a$ die aktuell vorherrschende Außentemperatur, $b$ das Gewicht der gezähmten Hauskatze in Kilogramm und $c$ die gefühlte Motivation  auf einer Skala von 1 -- 10 ist, diese sinnlose Aufgabe zu vollenden. \parencite[2]{noauthor_neue_nodate}\todo{Stimmt diese Formel wirklich? Ggf. weitere Quellen untersuchen}\par
Dieses Verhalten wird auch in Abbildung \ref{fig:cat01} veranschaulicht.

\begin{figure}[h]
    \centering
    \includegraphics[scale=.1]{"cat"}
    \caption{Nahaufnahme einer Katze, welche repräsentativ den Zustand geschlossener Augenlider für Säugetiere darstellt. \parencite{noauthor_neue_nodate}}
    \label{fig:cat01}
\end{figure}

\begin{table}[h]
    \centering
    \begin{tabular}{|c|c|c|c|c|c|c|c|}
        \hline
        $x$ & $-3$ & $-2$ & $-1$ & $0$ & $1$ & $2$ & $3$ \\
        \hline
        $y=x^2$ & $9$ & $4$ & $1$ & $0$ & $1$ & $4$ & $9$ \\
        \hline
    \end{tabular}
    \caption{Wertepaare der Funktion $y=x^2$ für $[-3;3]$}
    \label{tab:xSquare}
\end{table}


\newpage
%%% Abschluss des Dokuments %%%

\printbibheading[title={Literaturverzeichnis}]
\printbibliography[heading=subbibliography, title={Öffentliche Quellen}, notkeyword=intern]
\printbibliography[heading=subbibliography, title={Unternehmensinterne Quellen}, keyword=intern]

\listoffigures
\listoftables

\section*{Ehrenwörtliche Erklärung} % Text direkt aus der Vorgabe
Hiermit versichere ich, dass die vorliegende Arbeit von mir selbstständig und ohne unerlaubte Hilfe angefertigt worden ist, insbesondere, dass alle Stellen, die wörtlich oder sinngemäß aus Veröffentlichungen entnommen sind, durch Zitate als solche kenntlich gemacht wurden. Diese Versicherung bezieht sich auch auf die in der Arbeit verwendete bildliche Darstellungen, Tabellen, Zeichnungen, Skizzen, graphischen Darstellungen und dergleichen sowie auch für die Verwendung von text- oder codegenerierenden KI-Werkzeugen als Quelle.\par
Die Arbeit hat in gleicher oder ähnlicher Form noch keiner Prüfungsbehörde vorgelegen und ist nicht veröffentlicht. Sie wurde nicht, auch nicht auszugsweise, für eine andere Prüfungs- oder Studienleistung verwendet.\par
Ich bin damit einverstanden, dass die Arbeit einer elektronischen Plagiatsprüfung unterzogen werden kann. Die Regelungen der Prüfungsordnung zur Täuschung habe ich zur Kenntnis genommen.\par

\vspace{1cm}
\sigline{5cm}{Ort, Datum} \hfill \sigline{5cm}{Unterschrift}

\end{document}
