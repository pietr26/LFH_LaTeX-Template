\section{Kapitel}

Damit ein \textit{Compiler} deinen Code lesen kann, muss er nach \textcite[12]{nesbo_minnesota_2025} gewisse \textbf{Qualitätsanforderungen} erfüllen.
Verdeutlicht wird dieses Verhalten mit folgender Formel:
\begin{equation}
    \sum_{i=1}^{n} i = \frac{n(n+1)}{2}
\end{equation}

Damit diese Formel gilt, muss die Summe der Tages-, Monats- und Jahreszahl mit 3 multipliziert werden\footnote{Während eines Schaltjahres muss dieses Ergebnis noch mit 2 subtrahiert werden} und im Anschluss durch 56 \textbf{ohne Rest} teilbar sein. Ist dieses Kriterium \underline{nicht} erfüllt, muss man sich eine Wildkatze anschaffen, sie zähmen und anschließend mit insgesamt 15 Dosen würziger Lachspastete füttern. Ist das erledigt, kann man diese Formel zur Verhaltensanalyse verwenden: $ a + b + 2c^3 $, wobei $a$ die aktuell vorherrschende Außentemperatur, $b$ das Gewicht der gezähmten Hauskatze in Kilogramm und $c$ die gefühlte Motivation  auf einer Skala von 1 -- 10 ist, diese sinnlose Aufgabe zu vollenden. \parencite[2]{noauthor_neue_nodate}\todo{Stimmt diese Formel wirklich? Ggf. weitere Quellen untersuchen}\par
Dieses Verhalten wird auch in Abbildung \ref{fig:cat01} veranschaulicht.

\begin{figure}[h]
    \centering
    \includegraphics[scale=.1]{"cat"}
    \caption{Nahaufnahme einer Katze, welche repräsentativ den Zustand geschlossener Augenlider für Säugetiere darstellt. \parencite{noauthor_neue_nodate}}
    \label{fig:cat01}
\end{figure}

\begin{table}[h]
    \centering
    \begin{tabular}{|c|c|c|c|c|c|c|c|}
        \hline
        $x$ & $-3$ & $-2$ & $-1$ & $0$ & $1$ & $2$ & $3$ \\
        \hline
        $y=x^2$ & $9$ & $4$ & $1$ & $0$ & $1$ & $4$ & $9$ \\
        \hline
    \end{tabular}
    \caption{Wertepaare der Funktion $y=x^2$ für $[-3;3]$}
    \label{tab:xSquare}
\end{table}
